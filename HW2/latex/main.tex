\documentclass[UTF8, a4paper, 12pt]{ctexart}

% 基本设置
\usepackage[margin=1in]{geometry} % 页边距
\usepackage{amsmath,  amssymb,  amsthm} % 数学环境
\usepackage{enumitem} % 控制 enumerate 格式
\usepackage{lipsum}  % 测试文字, 可以删掉
\usepackage{float}
\usepackage{bbm}
\usepackage{listings}
\usepackage{color}
\usepackage{booktabs}
\usepackage{graphicx}
\usepackage{url} % 支持URL链接
\usepackage[american]{babel} % 保证英文单词断行按音节切割
\usepackage{microtype} % 调整字间距, 改善排版效果
\usepackage{fancyhdr} % 添加页眉页脚包

% 设置页面样式
\pagestyle{fancy}
\fancyhf{} % 清除所有页眉页脚
\fancyfoot[C]{\thepage} % 在页脚中央显示页码
\renewcommand{\headrulewidth}{0pt} % 去掉页眉分割线

% 重新定义 subsection* 命令, 使后面段落自动缩进
\makeatletter
\newcommand{\subsectionstar}[1]{%
\subsection*{#1}%
 \setlength{\parindent}{2em}% 设置段落缩进为2em
 \par\nobreak% 防止分页
}
\makeatother

% 让 enumerate 的序号对齐美观
\setlist[enumerate, 1]{label=\arabic*., leftmargin=1.5em,  labelsep=0.5em,  align=left}

% 标题部分
\title{偏微分方程数值解: 第二次作业}
\author{林立康 \quad 25210180078}
\date{\today}

\begin{document}
\maketitle

相关代码可从 \url{https://github.com/LiHuaGz/Spectral_Method.git} 获取.
\subsection*{1.(a)}
\begin{equation*}
  D_{N}(-x)=\frac{\sin(-(N+\frac{1}{2})x)}{\sin(-\frac{x}{2})}=\frac{\sin((N+\frac{1}{2})x)}{\sin(\frac{x}{2})}=D_{N}(x).
\end{equation*}

\begin{equation*}
  \begin{aligned}
  D_{N}(\frac{1}{2}+x) & =\frac{\sin((N+\frac{1}{2})(\frac{1}{2}+x))}{\sin(\frac{1}{2}(\frac{1}{2}+x))} \\
& =\frac{\sin((N+\frac{1}{2})\cdot\frac{1}{2})\cos((N+\frac{1}{2})x)+\cos((N+\frac{1}{2})\cdot\frac{1}{2})\cdot\sin((N+\frac{1}{2})x)}{\sin(\frac{1}{2}\cdot\frac{1}{2})\cdot\cos(\frac{1}{2}x)+\cos(\frac{1}{2}\cdot\frac{1}{2})\sin(\frac{1}{2}x)} \\
& =\frac{\sin((N+\frac{1}{2})\cdot\frac{1}{2})\cos(-(N+\frac{1}{2})x)+\cos((N+\frac{1}{2})\cdot\frac{1}{2})\cdot\sin(-(N+\frac{1}{2})x)}{\sin(\frac{1}{2}\cdot\frac{1}{2})\cdot\cos(-\frac{1}{2}x)+\cos(\frac{1}{2}\cdot\frac{1}{2})\sin(-\frac{1}{2}x)} \\
& =D_{N}(\frac{1}{2}-x).
\end{aligned}
\end{equation*}

\subsection*{1.(b)}
\begin{equation*}
  \begin{aligned}
\int_{0}^{2\pi}D_{N}(x)dx= & \int_{0}^{2\pi}(1+2\sum_{k=1}^{N}\cos kx)dx \\
 & =\int_{0}^{2\pi}dx+2\sum_{k=1}^{N}\int_{0}^{2\pi}\cos kx  dx \\
 & =2\pi.
\end{aligned}
\end{equation*}

\begin{equation*}
  \begin{aligned}
\int_{0}^{2\pi}D_{N}^{2}(x)dx & =\int_{0}^{2\pi}(1+2\sum_{k=1}^{N}\cos kx)^{2}dx \\
 & =\int_{0}^{2\pi}dx+4\int_{0}^{2\pi}\sum_{k=1}^{N}\cos kxdx+4\int_{0}^{2\pi}(\sum_{k=1}^{N}\cos kx)^{2}dx \\
 & =2\pi+4\sum_{j,k=1}^{N}\int_{0}^{2\pi}\cos jx\cdot\cos kxdx \\
 & =2\pi+4\sum_{k=1}^{N}\int_{0}^{2\pi}\cos^{2}kx dx \\
 & =2\pi+4\sum_{k=1}^{N}\int_{0}^{2\pi}\frac{\cos2kx+1}{2}dx \\
 & =2\pi+4N\pi.
\end{aligned}
\end{equation*}

因此, $$\frac{1}{2\pi}\int_{0}^{2\pi}D_{N}(x)dx=1, \quad \frac{1}{2\pi}\int_{0}^{2\pi}D_{N}^{2}(x)dx=2N+1.$$

\subsection*{1.(c)}
当$x \in [0, \pi]$时, $\sin\left(\frac{x}{2}\right)\geq\frac{2}{\pi}\cdot\frac{x}{2}=\frac{x}{\pi}$, 故
\begin{equation*}
  |D_N(x)|=\frac{\left|\sin((N+\frac{1}{2})x)\right|}{\sin(x/2)}\leq\frac{1}{\sin(x/2)}\leq\frac{\pi}{x}, \quad x \in (0, \pi].
\end{equation*}

可知$\int_0^{2\pi}\left|D_N(x)\right|dx=2\int_0^{\pi}\left|D_N(x)\right|dx$, 则
\begin{equation*}
\begin{aligned}
\int_0^{2\pi}\left|D_N(x)\right|dx & =2\int_0^\pi\left|D_N(x)\right|dx \\
 & \leq2\left(\int_0^{\frac{1}{2N+1}}(2N+1)dx+\int_{\frac{1}{2N+1}}^\pi\frac{\pi}{x}dx\right) \\
 & =2\left((2N+1)\cdot\frac{1}{2N+1}+\pi\ln\frac{\pi}{\frac{1}{2N+1}}\right) \\
 & =2\left(1+\pi\ln(2N+1)\right)  \\
 & \leq C\ln N, \quad N \geq 2,
\end{aligned}
\end{equation*}
其中$C$为常数.

\subsection*{1.(d)}
\begin{equation*}
  \begin{aligned}
|\phi(x)| & \leq\frac{1}{2\pi}\int_{0}^{2\pi}|\phi(t)||D_{N}(x-t)|dt \\
&\leq\frac{1}{2\pi}(\int_{0}^{2\pi}|\phi(t)|^{2}dt)^{\frac{1}{2}}\cdot(\int_{0}^{2\pi}|D_{N}(x-t)|^{2}dt)^{\frac{1}{2}} \\
&\leq \frac{1}{\sqrt{2\pi}}( \frac{1}{2\pi}\int_{0}^{2\pi}|\phi(t)|^{2}dt)^{\frac{1}{2}}\cdot(\int_{0}^{2\pi}|D_{N}(x-t)|^{2}dt)^{\frac{1}{2}} \\
 & =\frac{1}{\sqrt{2\pi}}||\phi||\sqrt{2\pi(2N+1)} \\
 & =\sqrt{2N+1}||\phi||,\quad x \in [0,2\pi].
\end{aligned}
\end{equation*}

其中, $\int_{0}^{2\pi}|D_{N}(x-t)|^{2}dt=\int_{x-2\pi}^{x}|D_{N}(y)|^{2}dy=2\pi(2N+1)$.

因此,
\begin{equation*}
||\phi||_{\infty}=\max_{x\in[0,2\pi]}|\phi(x)| \leq \sqrt{2N+1}||\phi||.
\end{equation*}


\subsection*{1.(e)}
可知,

\begin{equation*}
  \begin{aligned}
\sum_{N=1}^{M-1}D_{N}(x) & =\frac{1}{\sin\frac{x}{2}}\sum_{N=1}^{M-1}\sin((N+\frac{1}{2})x) \\
 & =\frac{1}{2i\sin\frac{x}{2}}\sum_{N=1}^{M-1}(e^{i(Nx+\frac{x}{2})}-e^{-i(Nx+\frac{x}{2})}),
 \end{aligned}
\end{equation*}

当$x \notin \pi\mathbb{Z}$时,
\begin{equation*}
\begin{aligned}
\sum_{N=1}^{M-1}(e^{i(Nx+\frac{x}{2})}-e^{-i(Nx+\frac{x}{2})}) &=e^{i\frac{x}{2}}\sum_{n=1}^{M-1}e^{iNx}-e^{-i\frac{x}{2}}\sum_{n=1}^{M-1}e^{-iNx} \\
 & = e^{i\frac{x}{2}}\cdot\frac{e^{ix}-e^{iMx}}{1-e^{ix}}-e^{-i\frac{x}{2}}\cdot\frac{e^{-ix}-e^{-iMx}}{(1-e^{-ix})} \\
 & =\frac{e^{i\frac{x}{2}}}{1-e^{ix}}(e^{ix}-e^{iMx})+\frac{e^{i\frac{x}{2}}}{1-e^{ix}} (e^{-ix}-e^{-iMx}) \\
 & =\frac{e^{i\frac{x}{2}}}{1-e^{ix}}[e^{ix}+e^{-ix}-(e^{iMx}+e^{-iMx})] \\
 & =\frac{1}{e^{-i\frac{x}{2}}-e^{i\frac{x}{2}}}[e^{ix}+e^{-ix}-(e^{iMx}+e^{-iMx})] \\
 & =\frac{1}{2i\sin(-\frac{x}{2})}(2\cos x-2\cos Mx),
\end{aligned}
\end{equation*}

所以
\begin{equation*}
\begin{aligned}
 \sum_{N=1}^{M-1}D_{N}(x) & =\frac{1}{\sin^2\frac{x}{2}}\cdot\frac{(1-2\sin^{2}\frac{x}{2})-(1-2\sin^{2}\frac{Mx}{2})}{2} \\
 & =\frac{1}{\sin^{2}\frac{x}{2}}(-\sin^{2}\frac{x}{2}+\sin^{2}\frac{Mx}{2}) \\
 & =-1+\frac{\sin^{2}\frac{Mx}{2}}{\sin^{2}\frac{x}{2}}.
\end{aligned}
\end{equation*}

从而
\begin{equation*}
\frac{1}{M}\sum_{N=0}^{M-1}D_{N}(x) = \frac{\sin^{2}\frac{Mx}{2}}{M\sin^{2}\frac{x}{2}}, \quad x \notin \pi\mathbb{Z}.
\end{equation*}

当 $x \in \pi\mathbb{Z}$时, 结论显然也成立.

\subsection*{2.(a)}

\textbf{(1) 理论分析}

易知$f(x)=e^{\sin(8x)}(-8\cos(x)\cos(8x)+64(2+\sin(x))(\sin(8x)-\cos^2(8x))+\sin^2(x))$.

对$[a,b]$做划分: $a=x_0<x_1<\cdots<x_N=b$, 其中$x_k=\frac{(b-a)k}{N}+a, k=0,1,\cdots,N$. 将$u(x)$做周期延拓, 并记$v(x)=p(x)u'(x)$.  则原方程可写为:
\begin{equation*}
  -v'(x)+q(x)u(x)=f(x), \quad x \in (a,b),
\end{equation*}
且$v(a)=v(b), u(a)=u(b)$.

$v'(x_k)$的二阶有限差分格式为:
\begin{equation*}
  v'(x_k)=\frac{v(x_k+\frac{h}{2})-v(x_k-\frac{h}{2})}{h}+O(h^2),\quad k=1,\cdots,N-1.
\end{equation*}

由于
\begin{equation*}
  \begin{aligned}
  v(x_k+\frac{h}{2})&=p(x_k+\frac{h}{2})u'(x_k+\frac{h}{2}) = p(x_k+\frac{h}{2})\cdot\frac{u(x_{k+1})-u(x_k)}{h}+O(h^2),\\
  v(x_k-\frac{h}{2})&=p(x_k-\frac{h}{2})u'(x_k-\frac{h}{2}) = p(x_k-\frac{h}{2})\cdot\frac{u(x_{k})-u(x_{k-1})}{h}+O(h^2),
  \end{aligned}
\end{equation*}
故
\begin{equation*}
  v'(x_k) \approx \frac{p(x_k+\frac{h}{2})\cdot\frac{u(x_{k+1})-u(x_k)}{h}-p(x_k-\frac{h}{2})\cdot\frac{u(x_{k})-u(x_{k-1})}{h}}{h},\quad k=0,1,\cdots,N-1.
\end{equation*}
将上述格式代入原方程, 可得线性方程组:
\begin{equation*}
  -\frac{p(x_k+\frac{h}{2})\cdot\frac{u(x_{k+1})-u(x_k)}{h}-p(x_k-\frac{h}{2})\cdot\frac{u(x_{k})-u(x_{k-1})}{h}}{h}+q(x_k)u(x_k)=f(x_k),\quad k=0,1,\cdots,N-1.
\end{equation*}

整理得:
\begin{equation*}
 \left\{\begin{matrix}
  -\frac{p_{k-\frac{1}{2}}}{h^2}u_{k-1}+\left(\frac{p_{k+\frac{1}{2}}+p_{k-\frac{1}{2}}}{h^2}+q_k\right)u_k-\frac{p_{k+\frac{1}{2}}}{h^2}u_{k+1}=f_k, \quad k=0,1,\cdots,N-1,\\
  u_N=u_0, \quad u_{-1}=u_{N-1},
  \end{matrix}
 \right. 
\end{equation*}
其中$p_{k\pm\frac{1}{2}}=p(x_k\pm\frac{h}{2}), q_k=q(x_k), f_k=f(x_k), u_k=u(x_k)$.

\textbf{(2) 数值结果}

$h$与$L^2$、$H^1$误差的关系如图\ref{fig:Q2_FDM1}所示. 由图可见, $L^2$ 误差(斜率: 2.02)和 $H^1$ 误差(斜率: 1.94)的拟合直线斜率均非常接近2, 这表明算法的收敛阶为2, 与理论分析结果一致.

\begin{figure}[H]
  \centering
  \includegraphics[width=0.75\textwidth]{./figures/Q2a_FDM_Error_Analysis.png}
  \caption{有限差分法$h$与$L^2$、$H^1$误差的关系}
  \label{fig:Q2_FDM1}
\end{figure}

\subsection*{2.(b)}
\textbf{(1) Fourier-Galerkin谱方法}

记$X_N=span\left\{e^{ikx}: k=-N/2, \cdots, N/2-1\right\}, u_N=\sum_{k=-N/2}^{N/2-1}\tilde{u_k} e^{ikx}$, 则Fourier-Galerkin谱方法即找出$u_N \in X_N$, 使得
\begin{equation}
  \int_{0}^{2\pi} [- (p(x)u_N'(x))' + q(x)u_N(x) - f(x)] e^{-imx} dx = 0, \quad m=-N/2,\cdots,N/2-1.
  \label{eq:T2_1}
\end{equation}

由于
\begin{equation*}
  \begin{aligned}
    &\int_{0}^{2\pi} (p(x)u_N'(x))' e^{-imx} dx \\
    &= p(x)u_N'(x)e^{-imx}\bigg|_{0}^{2\pi} - \int_{0}^{2\pi} p(x)u_N'(x)(-ime^{-imx}) dx \\
    &= im\int_{0}^{2\pi} p(x)u_N'(x)e^{-imx} dx\\
    &= im\int_{0}^{2\pi} p(x)\left(\sum_{k=-N/2}^{N/2-1}ik\tilde{u_k}e^{ikx}\right)e^{-imx} dx \\
    &= \sum_{k=-N/2}^{N/2-1}-mk\tilde{u_k}\int_{0}^{2\pi} p(x)e^{i(k-m)x} dx, \\
    &= \sum_{k=-N/2}^{N/2-1}-2\pi mk\tilde{u_k}\hat{p}_{m-k},
  \end{aligned}
\end{equation*}
且
\begin{equation}
  \left\{
    \begin{matrix}
    \begin{aligned}
    &\int_{0}^{2\pi} q(x)u_N(x)e^{-imx} dx = 2\pi \sum_{k=-N/2}^{N/2-1}\hat{q}_{m-k}\tilde{u}_k,\\
    &\int_{0}^{2\pi} f(x)e^{-imx} dx = 2\pi\hat{f}_m,
    \end{aligned}
    \end{matrix}
  \right.
  \label{eq:T2_2}
\end{equation}
故原问题可化为求解线性方程组:
\begin{equation}
  \sum_{k=-N/2}^{N/2-1}mk\hat{p}_{m-k}\tilde{u}_k+\sum_{k=-N/2}^{N/2-1}\hat{q}_{m-k}\tilde{u}_k=\hat{f}_m, \quad m=-N/2,\cdots,N/2-1.
  \label{eq:T2_3}
\end{equation}

\textbf{(2) Fourier-Galerkin伪谱方法}

事实上, 式(\ref{eq:T2_1})也即
\begin{equation*}
  -\widehat{(p(x)u_N'(x))'}_m + \widehat{q(x)u_N(x)}_m = \hat{f}_m,\quad m=-N/2,\cdots,N/2-1,
\end{equation*}
由式(\ref{eq:T2_3})可知, 这可以视为作用在$\tilde{u}=\left[\tilde{u}_{-N/2}, \cdots, \tilde{u}_{N/2-1}\right]^T$上的线性算子. 将等式左端第一项和第二项对应的线性算子分别记为$\mathcal{L}_1, \mathcal{L}_2$, 则Fourier-Galerkin伪谱方法即找出$\tilde{u}$使得
\begin{equation}
  \mathcal{L}_1(\tilde{u}) + \mathcal{L}_2(\tilde{u}) = \hat{f}.
  \label{eq:T2_4}
\end{equation}

若直接使用式(\ref{eq:T2_3})左端的卷积形式计算$\mathcal{L}_1(\tilde{u})$, 则每次计算都需要$O(N^2)$的时间复杂度. 为了提高计算效率, 使用伪谱方法计算$\mathcal{L}_1(\tilde{u})$. 

使用高斯消元法时的计算流程如下:
(1) 设FFT和IFFT对应矩阵分别为$\mathbf{F}, \mathbf{F}^{-1}$, 设$P=diag(p(x_0), \cdots, p(x_{N-1})), Q=diag(q(x_0), \cdots, q(x_{N-1}))$, 
(2) 计算$\mathcal{L}_1$算子对应的矩阵$$\mathbf{L_1} = -i\mathbf{m} \cdot \mathbf{F} \cdot P \cdot \mathbf{F}^{-1} \cdot i\mathbf{m},$$
其中$\mathbf{m}= (0, 1, ..., N/2-1, 0, -N/2+1, ..., 0)$;
(3) 计算$\mathcal{L}_2$算子对应的矩阵$$\mathbf{L_2} = \mathbf{F} \cdot Q \cdot \mathbf{F}^{-1};$$
(4) 求解线性方程组$ (-i\mathbf{m} \cdot \mathbf{F} \cdot P \cdot \mathbf{F}^{-1} \cdot i\mathbf{m} + \mathbf{F} \cdot Q \cdot \mathbf{F}^{-1}) \tilde{u} = \hat{f}.$

使用共轭梯度法求解时的计算流程如下:
(1) 设当前迭代点的$\tilde{u}_N$已知, 通过IFFT计算$u_N'(x_j)$, 其中$x_j=\frac{2\pi j}{N}, j=0,1,\cdots,N-1$为配点;
(2) 逐点相乘得到$p(x_j)u_N'(x_j)$;
(3) 通过FFT计算$p(x)u_N'(x)$的傅里叶系数, 并对其乘以$i\mathbf{m}$得到$-(p(x)u_N'(x))'$的傅里叶系数.
$\mathcal{L}_2(\tilde{u})$的计算类似.

\textbf{(3) 数值结果}

使用高斯消元法求解时, $N$与$L^2$误差的关系如图\ref{fig:Q2_spectral_GE}所示. 可见$L^2$ 误差随 $N$ 增加而呈指数级快速下降, 在 $N \approx 500$ 之后,误差达到了约 $10^{-12}$ 量级, 这代表了由双精度浮点数运算所决定的机器精度极限. 该结果验证了 Fourier 伪谱方法的谱精度. 随后误差略有上升, 这可能是由累积误差所致.

\begin{figure}[H]
  \centering
  \includegraphics[width=0.75\textwidth]{./figures/Q2b_F_spectral_Error_Analysis.png}
  \caption{伪谱方法$N$与$L^2$误差的关系(高斯消元法)}
  \label{fig:Q2_spectral_GE}
\end{figure}


\subsection*{2.(c)}
使用共轭梯度法求解(容忍精度$10^{-10}$), $N$与$L^2$误差的关系如图\ref{fig:Q2_spectral_CG}所示. 从图中可以观测到:$L^2$误差从 $N=100$ 时的 $10^{-1}$ 量级迅速降至 $N \approx 250$ 时的 $10^{-7}$ 量级以下, 这表明该伪谱方法对于求解光滑周期解问题具有谱精度; 当 $N$ 增加到约500之后, $L^2$ 误差不再下降, 这可能是由于$\mathcal{L}_1+\mathcal{L}_2$不是自伴算子, 从而共轭梯度法不能保证收敛; 后面的误差上升可能是由累积误差所致(也可能是我的实现方式不对?).

达到10位精度时共轭梯度法所需循环次数与$N$的关系如图\ref{fig:Q2_spectral_iter}所示. 由图可知, CG迭代次数与 $N$ 近似成正比(例如 $N=1000$ 时迭代约1000次,$N=4000$ 时迭代约4000次), 这说明共轭梯度法并没有因为提前达到10位精度而提前终止, 应证了上述``可能由于$\mathcal{L}_1+\mathcal{L}_2$不是自伴算子, 导致使用共轭梯度法不能保证收敛".


\begin{figure}[H]
  \centering
  \includegraphics[width=0.75\textwidth]{./figures/Q2c_F_spectral_CG_Iterations.png}
  \caption{共轭梯度法所需循环次数与$N$的关系}
  \label{fig:Q2_spectral_iter}
\end{figure}

\begin{figure}[H]
  \centering
  \includegraphics[width=0.75\textwidth]{./figures/Q2c_F_spectral_Error_Analysis_CG.png}
  \caption{伪谱方法$N$与$L^2$误差的关系(共轭梯度法)}
  \label{fig:Q2_spectral_CG}
\end{figure}

\subsection*{2.(d)}
\textbf{(1) 理论分析}

Fourier配点法即要找近似解 $u_N(x)=\sum_{k=-N/2}^{N/2-1}\tilde{u_k} e^{ikx}$, 使得在所有配点 $x_j$ 上原微分方程成立:
\begin{equation*}
  -p'(x_j)u'_N(x_j) - p(x_j)u''_N(x_j) + q(x_j)u_N(x_j) = f(x_j), \quad j=0, 1, \dots, N-1.
\end{equation*}

记$\mathbf{u} = [u(x_0), \dots, u(x_{N-1})]^T$为在配点 $x_j$ 上的函数值向量, $\mathbf{f} = [f(x_0), \dots, f(x_{N-1})]^T, \mathbf{P} = diag(p(x_0), \dots, p(x_{N-1})), 
\mathbf{P'} = diag(p'(x_0), \dots, p'(x_{N-1})), \mathbf{Q} = diag(q(x_0), \dots, q(x_{N-1}))$. 记一阶和二阶微分矩阵分别为 $D^{(1)}$ 和 $D^{(2)}$, 则原问题可表示为矩阵形式:
\begin{equation*}
  (-\mathbf{P'} D^{(1)}  - \mathbf{P} D^{(2)}  + \mathbf{Q}) \mathbf{u} = \mathbf{f}.
\end{equation*}

\textbf{(2) 数值结果}

使用高斯消元法求解时, $N$与$L^2$误差的关系如图\ref{fig:Q2d_F_collocation_Error_Analysis_GE}所示. 可见$L^2$ 误差随 $N$ 增加而呈指数级快速下降, 在 $N \approx 500$ 之后,误差达到了 $10^{-11}$ 量级, 这代表了由双精度浮点数运算所决定的机器精度极限. 该结果验证了 Fourier 配点法的谱精度. 同时说明了在该题中配点法的精度可能不如上述的Galerkin伪谱方法.
\begin{figure}[H]
  \centering
  \includegraphics[width=0.75\textwidth]{./figures/Q2d_F_collocation_Error_Analysis_GE.png}
  \caption{配点法$N$与$L^2$误差的关系(高斯消元法)}
  \label{fig:Q2d_F_collocation_Error_Analysis_GE}
\end{figure}

\subsection*{2.(e)}
达到10位精度时共轭梯度法所需循环次数与$N$的关系如图\ref{fig:Q2e_F_collocation_CG_Iterations}所示.
使用共轭梯度法求解时, $N$与$L^2$误差的关系如图\ref{fig:Q2d_F_collocation_Error_Analysis_CG}所示.  图\ref{fig:Q2d_F_collocation_Error_Analysis_CG}中, 初期误差的指数下降证明了方法的谱精度, 而最终误差趋向于$10^{-5}$ , 高于图\ref{fig:Q2d_F_collocation_Error_Analysis_GE}中的机器精度极限 ($10^{-11}$), 可能是因为求解时使用了共轭梯度法, 而$-\mathbf{P'} D^{(1)}  - \mathbf{P} D^{(2)}  + \mathbf{Q}$不是厄米特阵($D^{(1)}$不是对称阵), 导致共轭梯度法不能保证收敛到机器精度.

\begin{figure}[H]
  \centering
  \includegraphics[width=0.75\textwidth]{./figures/Q2e_F_collocation_CG_Iterations.png}
  \caption{共轭梯度法所需循环次数与$N$的关系}
  \label{fig:Q2e_F_collocation_CG_Iterations}
\end{figure}

\begin{figure}[H]
  \centering
  \includegraphics[width=0.75\textwidth]{./figures/Q2d_F_collocation_Error_Analysis_CG.png}
  \caption{配点法$N$与$L^2$误差的关系(共轭梯度法)}
  \label{fig:Q2d_F_collocation_Error_Analysis_CG}
\end{figure}

\subsection*{3.}
\textbf{(1) 理论分析}

将解 $u(x,t)$ 展开为截断的傅里叶级数:$$u(x, t) \approx u_N(x, t) = \sum_{k=-N/2}^{N/2-1} \hat{u}_k(t) e^{ikx},$$
记$$R_N(x,t) = \frac{\partial u_N}{\partial t} - \epsilon \frac{\partial^2 u_N}{\partial x^2} - u_N \frac{\partial u_N}{\partial x}, $$
原问题化为找$u_N$, 使得$$\int_0^{2\pi} R_N(x, t) \cdot e^{-imx} \,dx = 0, \quad  m = -N/2, \dots, N/2-1,$$
也即找$u\in X_N=span\{e^{ikx}: k=-N/2, \cdots, N/2-1\}$,使得
\begin{equation}
\frac{d\hat{u}_k}{dt} =  \epsilon \widehat{\left(\frac{\partial^2 u}{\partial x^2}\right)}_k + \widehat{\left(u \frac{\partial u}{\partial x}\right)}_k, \quad k=-N/2, \dots, N/2-1,
\label{eq:T3_1}
\end{equation}
这里的 $\widehat{(\cdot)}_k$ 表示“括号内项的第$k$个傅里叶系数”. 记 $\hat{N}_k = \widehat{\left(u \frac{\partial u}{\partial x}\right)}_k$ , 则上式可写为
\begin{equation}
  \frac{d\hat{u}_k}{dt} = -\epsilon k^2 \hat{u}_k + \hat{N}_k, \quad k=-N/2, \dots, N/2-1.
  \label{eq:T3_2}
\end{equation}


计算非线性项 $\hat{N}_k = \widehat{\left(u \frac{\partial u}{\partial x}\right)}_k$ 的流程:(1)通过FFT计算 $\frac{\partial u}{\partial x}$, (2)逐点相乘$u .*\frac{\partial u}{\partial x} $, (3)再次使用FFT计算乘积的傅里叶系数.

注: 实际计算使用的是式(\ref{eq:T3_1})的转化形式: 
\begin{equation}
  \frac{d\tilde{u}^{fft}_k}{dt} = \widetilde{(\epsilon \frac{\partial^2 u}{\partial x^2} + u \frac{\partial u}{\partial x})}^{fft}_k, \quad k=-N/2, \dots, N/2-1.
  \label{eq:T3_3}
\end{equation}
流程: (1) 已知$t=t_n$时刻的$\tilde{u}^{fft}_k(t_n)$, 通过IFFT计算 $u(x_j, t_n)$ , $\frac{\partial u}{\partial x}(x_j, t_n)$和$\frac{\partial^2 u}{\partial x^2}(x_j, t_n)$; (2) 逐点相乘得到 $u \frac{\partial u}{\partial x}(x_j, t_n)$; (3) 通过FFT计算式(\ref{eq:T3_3})右端的傅里叶系数; (4) 使用四阶Runge-Kutta法求解ODE系统, 得到$t=t_{n+1}$时刻的$\tilde{u}^{fft}_k(t_{n+1})$.

\textbf{(2) 数值结果}

$t=1$时刻的数值结果如图\ref{fig:Q3_t1}所示, 其中$\epsilon=0.03, N_x=N_t=128$. $(x,t)$与$u(x,t)$的关系如图\ref{fig:Q3_3d}所示.

\begin{figure}[H]
  \centering
  \includegraphics[width=0.75\textwidth]{./figures/Q3_burgers_solution.png}
  \caption{时刻$t=1$的数值解}
  \label{fig:Q3_t1}
\end{figure}

\begin{figure}[H]
  \centering
  \includegraphics[width=0.9\textwidth]{./figures/Q3_burgers_3d.png}
  \caption{$(x,t)$与$u(x,t)$的关系}
  \label{fig:Q3_3d}
\end{figure}


\end{document}