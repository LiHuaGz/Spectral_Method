\documentclass[UTF8, a4paper, 12pt]{ctexart}

% 基本设置
\usepackage[margin=1in]{geometry} % 页边距
\usepackage{amsmath,  amssymb,  amsthm} % 数学环境
\usepackage{enumitem} % 控制 enumerate 格式
\usepackage{lipsum}  % 测试文字, 可以删掉
\usepackage{float}
\usepackage{bbm}
\usepackage{listings}
\usepackage{color}
\usepackage{booktabs}
\usepackage{graphicx}
\usepackage{url} % 支持URL链接
%\usepackage[american]{babel} % 保证英文单词断行按音节切割
\usepackage{microtype} % 调整字间距, 改善排版效果
\usepackage{fancyhdr} % 添加页眉页脚包

% 设置页面样式
\pagestyle{fancy}
\fancyhf{} % 清除所有页眉页脚
\fancyfoot[C]{\thepage} % 在页脚中央显示页码
\renewcommand{\headrulewidth}{0pt} % 去掉页眉分割线

% 将图表标题改为中文
\ctexset{
  figurename=图,
  tablename=表
}

% 重新定义 subsection* 命令, 使后面段落自动缩进
\makeatletter
\newcommand{\subsectionstar}[1]{%
\subsection*{#1}%
 \setlength{\parindent}{2em}% 设置段落缩进为2em
 \par\nobreak% 防止分页
}
\makeatother

% 让 enumerate 的序号对齐美观
\setlist[enumerate, 1]{label=\arabic*., leftmargin=1.5em,  labelsep=0.5em,  align=left}

% 标题部分
\title{偏微分方程数值解: 第三次作业}
\author{林立康 \quad 25210180078}
\date{\today}

\begin{document}

% 确保图表标题使用中文
\renewcommand{\figurename}{图}
\renewcommand{\tablename}{表}

\maketitle

相关代码可从 \url{https://github.com/LiHuaGz/Spectral_Method.git} 获取.
\subsection*{1.(a)}

令 $\mathcal{L} u = \partial_x^m [(1-x^2)^m \partial_x^m u].$ 显然 $\forall u \in P_n, \mathcal{L}u \in P_n.$
对于 $\forall \phi \in P_{n-1}$, 我们考察内积:
\begin{align*}
(\mathcal{L} L_n, \phi)_{w^{0,0}} &= \int_{-1}^1 \partial_x^m [(1-x^2)^m \partial_x^m L_n] \phi \, dx \\
&= (-1)^m \int_{-1}^1 (1-x^2)^m \partial_x^m L_n \cdot \partial_x^m \phi \, dx \quad (\text{分部积分}) \\
&= (-1)^m \cdot (-1)^m \int_{-1}^1 L_n \partial_x^m [(1-x^2)^m \partial_x^m \phi] \, dx \\
&= (L_n, \mathcal{L}\phi)_{w^{0,0}}.
\end{align*}
由于 $\phi \in P_{n-1}$, 故 $\mathcal{L}\phi \in P_{n-1}$, 从而
$$ (L_n, \mathcal{L}\phi)_{w^{0,0}} = 0. $$
由 $I=(-1,1), w^{0,0} \equiv 1$ 对应正交多项式的唯一性, 必存在常数 $\tilde{\lambda}_{n,m}$ 使得:
$$ \mathcal{L} L_n = \tilde{\lambda}_{n,m} L_n. $$

\noindent \textbf{计算特征值 $\tilde{\lambda}_{n,m}$:}

考察方程两边的首项系数.
由 Leibniz 公式, $\mathcal{L} L_n = \sum_{k=0}^m \binom{m}{k} [(1-x^2)^m]^{(k)} (\partial_x^m L_n)^{(m-k)}.$
其首项系数由 $k=0$ 项贡献 (最高次项):
\begin{align*}
\text{Coeff}(\mathcal{L} L_n) &= \binom{m}{0} \cdot (-1)^m \cdot k_n \cdot \frac{(2m)!}{(2m-0)!} \cdot \frac{n!}{(n-(2m-0))!} \\
&= (-1)^m \cdot k_n \cdot (2m)! \sum_{k=0}^m \binom{m}{k} \binom{n}{2m-k} \\
&= (-1)^m \cdot k_n \cdot \frac{(n+m)!}{(n-m)!}.
\end{align*}
其中 $k_n$ 为 $L_n$ 的首项系数.
比较两边系数即得:
$$ \tilde{\lambda}_{n,m} = (-1)^m \frac{(n+m)!}{(n-m)!}. $$
从而原式成立.
\subsection*{1.(b)}

利用 (a) 中的算子性质:
\begin{align*}
\int_{-1}^1 L_n^{(m)}(x) L_k^{(m)}(x) (1-x^2)^m \, dx 
&= (-1)^m \int_{-1}^1 L_n(x) \partial_x^m \left[ L_k^{(m)}(x) (1-x^2)^m \right] \, dx \\
&= (-1)^m \int_{-1}^1 L_n(x) \mathcal{L} L_k(x) \, dx \\
&= (-1)^m \int_{-1}^1 L_n(x) \tilde{\lambda}_{k,m} L_k(x) \, dx \\
&= \frac{(n+m)!}{(n-m)!} \|L_n\|^2 \delta_{n,k}.
\end{align*}

\subsection*{1.(c)}

\textbf{$m=0$ 时:}
由递推关系 $(n+1)L_{n+1}(x) = (2n+1)xL_n(x) - nL_{n-1}(x), n \ge 1$, 易知:
$$ L_n(\pm 1) = (\pm 1)^n. $$

\textbf{$m \ge 1$ 时:} 
当 $x = -1$ 时, 
\begin{align*}
& \partial_x^m \left[ (1+x)^m (1-x)^m L_n^{(m)} \right] \Bigg|_{x=-1} \\
=& \sum_{k=0}^m \binom{m}{k} \left[ (1+x)^m \right]^{(k)} \left[ (1-x)^m L_n^{(m)} \right]^{(m-k)} \Bigg|_{x=-1} \\
=& m! \cdot 2^m \cdot L_n^{(m)}(-1).
\end{align*}
故由 (a),
$$ L_n^{(m)}(-1) = \frac{(-1)^m \tilde{\lambda}_{n,m} (-1)^n}{m! 2^m} = \frac{(-1)^{m+n}}{m! 2^m} \frac{(n+m)!}{(n-m)!}. $$

同理, 当 $x=1$ 时:
\begin{align*}
\partial_x^m \left[ (1-x)^m (1+x)^m L_n^{(m)} \right] \Bigg|_{x=1} 
&= (-1)^m \cdot m! \cdot 2^m L_n^{(m)}(1).
\end{align*}
结合 $\mathcal{L} L_n(1) = \tilde{\lambda}_{n,m} L_n(1) = (-1)^m \frac{(n+m)!}{(n-m)!}$, 解得:
$$ L_n^{(m)}(1) = \frac{\tilde{\lambda}_{n,m}}{m! 2^m} = \frac{1}{m! 2^m} \frac{(n+m)!}{(n-m)!}. $$

\subsection*{1.(d)}

\textbf{$n$ 为奇数时:}
由于 $L_n(-x) = (-1)^n L_n(x) = -L_n(x)$, 故 $L_n(0) = 0.$

\textbf{$n$ 为偶数时:}
由勒让德多项式的显式公式:
$$ L_n(x) = \frac{1}{2^n} \sum_{l=0}^{n/2} (-1)^l \frac{(2n-2l)!}{l! (n-l)! (n-2l)!} x^{n-2l}. $$
取 $x=0$, 仅当 $n-2l=0$ 即 $l=n/2$ 时有非零项:
$$ L_n(0) = \frac{1}{2^n} (-1)^{n/2} \frac{n!}{(n/2)! (n/2)! 0!} = \frac{(-1)^{n/2} n!}{2^n [(n/2)!]^2}. $$

\subsection*{1.(e)}

利用 Laplace 积分表示:
$$ L_n(x) = \frac{1}{\pi} \int_0^\pi \left( x + i \sqrt{1-x^2} \cos \theta \right)^n d\theta. $$
考察被积函数的模长平方:
\begin{align*}
\left| x + i \sqrt{1-x^2} \cos \theta \right|^2 
&= x^2 + (1-x^2)\cos^2 \theta \\
&= x^2 (1-\cos^2 \theta) + \cos^2 \theta \\
&= x^2 \sin^2 \theta + \cos^2 \theta \le \sin^2 \theta + \cos^2 \theta = 1.
\end{align*}

\textbf{当 $n \ge 2$ 时:}
$$ \left| x + i \sqrt{1-x^2} \cos \theta \right|^n \le \left| x + i \sqrt{1-x^2} \cos \theta \right|^2. $$
故:
\begin{equation*}
  |L_n(x)| \le \frac{1}{\pi} \int_0^\pi (x^2 \sin^2 \theta + \cos^2 \theta) \, d\theta \le \frac{1}{\pi} \left( x^2 \frac{\pi}{2} + \frac{\pi}{2} \right) = \frac{1+x^2}{2}.
\end{equation*}
当 $n=1$ 时, $L_1(x)=x$, $|x| \le \frac{1+x^2}{2}$ (即 $(|x|-1)^2 \ge 0$) 显然成立.

\subsection*{1.(f)}

\textbf{1. 首先考虑第三个式子:}

已知 $L_n(x) = \frac{1}{2n+1}(L'_{n+1}(x) - L'_{n-1}(x)).$
\begin{align*}
\partial_x^k u &= \sum_{n=0}^N \hat{u}_n^{(k)} L_n 
= \sum_{n=0}^N \hat{u}_n^{(k)} \frac{1}{2n+1} (L'_{n+1} - L'_{n-1}) \\
&= \sum_{n=1}^{N+1} \frac{\hat{u}_{n-1}^{(k)}}{2n-1} L'_n - \sum_{n=0}^{N-1} \frac{\hat{u}_{n+1}^{(k)}}{2n+3} L'_n \\
&= \sum_{n=1}^{N-1} \left[ \frac{\hat{u}_{n-1}^{(k)}}{2n-1} - \frac{\hat{u}_{n+1}^{(k)}}{2n+3} \right] L'_n + (\text{边界项}).
\end{align*}
另一方面, $\partial_x^k u = \partial_x (\partial_x^{k-1} u) = \sum_{n=0}^N \hat{u}_n^{(k-1)} L'_n.$
对比 $L'_n$ 的系数得:
$$ \hat{u}_n^{(k-1)} = \frac{1}{2n-1} \hat{u}_{n-1}^{(k)} - \frac{1}{2n+3} \hat{u}_{n+1}^{(k)}. $$

\textbf{2. 其次考虑第一个式子:}

将上式重写为:
$$ \frac{\hat{u}_{n-1}^{(1)}}{2n-1} = \hat{u}_n^{(0)} + \frac{\hat{u}_{n+1}^{(1)}}{2n+3}. $$
作变量替换 $n \to n+1$:
$$ \frac{\hat{u}_n^{(1)}}{2n+1} = \hat{u}_{n+1}^{(0)} + \frac{\hat{u}_{n+2}^{(1)}}{2n+5}. $$
这是一个关于 $\frac{\hat{u}_n^{(1)}}{2n+1}$ 的递归式.
展开得:
\begin{align*}
\frac{\hat{u}_n^{(1)}}{2n+1} &= \hat{u}_{n+1}^{(0)} + \left( \hat{u}_{n+3}^{(0)} + \frac{\hat{u}_{n+4}^{(1)}}{2n+9} \right) \\
&= \hat{u}_{n+1}^{(0)} + \hat{u}_{n+3}^{(0)} + \hat{u}_{n+5}^{(0)} + \dots \\
&= \sum_{\substack{p=n+1 \\ p+n \text{ odd}}}^N \hat{u}_p^{(0)}.
\end{align*}

\textbf{3. 最后考虑第二个式子:}

由上可知
$$ \hat{u}_n^{(2)} = (2n+1) \sum_{\substack{p=n+1 \\ p+n \text{ odd}}}^N \hat{u}_p^{(1)}. $$
将 $\hat{u}_p^{(1)} = (2p+1) \sum_{\substack{j=p+1 \\ j+p \text{ odd}}}^N \hat{u}_j^{(0)}$ 代入:
$$ \hat{u}_n^{(2)} = (2n+1) \sum_{\substack{p=n+1 \\ p+n \text{ odd}}}^N (2p+1) \sum_{\substack{j=p+1 \\ j+p \text{ odd}}}^N \hat{u}_j^{(0)}. $$
交换求和顺序:
外部求和指标 $p$ 从 $n+1$ 开始, 步长为 2. 内部指标 $j$ 从 $p+1$ 开始, 步长为 2.
这意味着 $j$ 的范围是从 $n+2$ 到 $N$, 且 $j$ 与 $n$ 同奇偶 (即 $j+n$ 为偶数).
对于固定的 $j$, $p$ 的范围是从 $n+1$ 到 $j-1$, 步长为 2.
\begin{align*}
\hat{u}_n^{(2)} &= (2n+1) \sum_{\substack{j=n+2 \\ j+n \text{ even}}}^N \left( \sum_{\substack{p=n+1 \\ \text{step}=2}}^{j-1} (2p+1) \right) \hat{u}_j^{(0)}.
\end{align*}
利用等差数列求和 $\sum_{k} (2k+1) = k(k+1) + C$:
中间的求和项为 $\frac{1}{2} [j(j+1) - n(n+1)].$ (基于 $p$ 的求和结果).
最终得到:
$$ \hat{u}_n^{(2)} = \frac{2n+1}{2} \sum_{\substack{j=n+2 \\ j+n \text{ even}}}^N [j(j+1) - n(n+1)] \hat{u}_j^{(0)}. $$

\subsection*{2.(a)(b)}

本文将正交多项式的根转化为对称三对角矩阵的特征值问题求解.
\begin{itemize}
    \item \textbf{LG格点}:为 $(N+1)$ 阶 Legendre 多项式 $P_{N+1}(x)$ 的零点.由递推关系构造 $(N+1)\times(N+1)$ 的三对角矩阵 $J_{LG}$,其非对角元素为 $\beta_i = \sqrt{\frac{i^2}{4i^2-1}}$.节点即为 $J_{LG}$ 的特征值.
    \item \textbf{LGL格点}:包含端点 $\{-1, 1\}$ 以及 $P'_N(x)$ 的 $N-1$ 个零点.内部节点对应于 Jacobi 多项式 $P_{N-1}^{(1,1)}(x)$ 的零点,对应矩阵 $J_{LGL}$ 的大小为 $(N-1)\times(N-1)$,非对角元素为 $\beta_i = \sqrt{\frac{i(i+2)}{(2i+1)(2i+3)}}$.
\end{itemize}

为了验证程序的正确性,将计算结果与 \texttt{numpy.polynomial.legendre} 标准库及通过多项式求导直接求根的结果进行了对比.在 $N=50$ 的情况下,最大绝对误差如下表所示,误差均在机器精度 ($10^{-15}$) 量级,验证了算法实现的准确性.

\begin{table}[H]
    \centering
    \caption{数值解与参考解的最大绝对误差 ($N=50$)}
    \begin{tabular}{lcc}
        \hline
        类型 & 节点误差 & 权重误差 \\
        \hline
        Legendre-Gauss (LG) & $3.33 \times 10^{-16}$ & $3.05 \times 10^{-15}$ \\
        Legendre-Gauss-Lobatto (LGL) & $3.11 \times 10^{-15}$ & $1.60 \times 10^{-16}$ \\
        \hline
    \end{tabular}
\end{table}

图 \ref{fig:nodes_comp} 展示了 $N=15$ 时 LG 与 LGL 节点的分布情况.
\begin{figure}[H]
    \centering
    \includegraphics[width=0.9\textwidth]{figures/Q2_LG_LGL_nodes.png}
    \caption{Legendre-Gauss (LG) 与 Legendre-Gauss-Lobatto (LGL) 节点分布对比 ($N=15$)}
    \label{fig:nodes_comp}
\end{figure}

\subsection*{3.(a)}
易知
$$u'(x) = L'_{N+1}(x) - L'_{N-1}(x) = (2N+1)L_N(x), $$
故
$$\|u'\|^2 = (2N+1)^2 \|L_N\|^2 = (2N+1)^2 \cdot \frac{2}{2N+1} = 2(2N+1),$$
即题中不等式右端$cN^{1/2}\|u'\| \sim c N^{1/2} \cdot N^{1/2} = cN.$ 现考虑不等式左端.

(1) 首先说明对于Legendre-Gauss-Lobatto (LGL) 插值, 题中结论不成立:

LGL 点 $\{x_j\}_{j=0}^N$ 是 $(1-x^2)L'_N(x)$ 的根, 故
$$u(x_j) = L_{N+1}(x_j) - L_{N-1}(x_j) = (1-x_j^2)L'_N(x_j) = 0, \quad \forall j=0, \dots, N.$$
由于 $I_N u$ 是一个次数 $\le N$ 的多项式,且在 $N+1$ 个点上为 0, 因此 $I_N u \equiv 0.$
此时
$$\|I_N u - u\|_{H^1} = \|-u\|_{H^1} = \|u\|_{H^1} = \|u\| + \|u'\| \sim \frac{1}{\sqrt{N}} + \sqrt{N},$$
因此题中不等式不成立.
(2) 其次, 如果 $I_N$ 是 Legendre-Gauss (LG) 插值, 则结论成立:

LG 点 $\{y_j\}_{j=0}^N$ 是 $L_{N+1}(x)$ 的根.因此:$$u(y_j) = L_{N+1}(y_j) - L_{N-1}(y_j) = -L_{N-1}(y_j), \quad j=0, \dots, N.$$
由于$deg(I_N u) \leq N, deg(-L_{N-1}(x))\leq N-1,$ 故
$$I_N u(x) = -L_{N-1}(x),$$
$$I_N u - u = -L_{N-1} - (L_{N+1} - L_{N-1}) = -L_{N+1}.$$
从而
$$
\|I_N u - u\|_{H^1} = \|L_{N+1}\| + \|L'_{N+1}\|.$$
前一项 $\|L_{N+1}\| \sim \frac{1}{\sqrt{N}}$ , 后一项由于
$$L'_{N+1} = \sum_{k=0, k+N+1 \text{ odd}}^{N} (2k+1)L_k,$$
$$\|L'_{N+1}\|^2 = \sum (2k+1)^2 \|L_k\|^2 = \sum (2k+1)^2 \frac{2}{2k+1} = \sum 2(2k+1) \approx 2 \sum_{k=1}^{N/2} 4k \approx 2N^2.$$
所以:$$\|I_N u - u\|_{H^1} \sim C \cdot N.$$

\subsection*{3.(b)}
以下证明过程与书上引理3.4类似.
令$ x = \cos \theta, \hat{u}(\theta) = u(\cos \theta).$ 由于$I_N u(\pm 1)=0$, 故$\frac{(I_N u(x))^2}{1-x^2}$是多项式, 因此
\begin{align*}
\| I_N u \|_{w^{-1, -1}}^2 &= \int_{-1}^1 \frac{(I_N u)^2}{1-x^2} \, dx = \sum_{j=0}^N \frac{(I_N u(x_j))^2}{1-x_j^2} w_j = \sum_{j=0}^N \frac{u^2(x_j)}{1-x_j^2} w_j = \sum_{j=1}^{N-1} \frac{u^2(x_j)}{1-x_j^2} w_j \\
&\lesssim \sum_{j=1}^{N-1} \frac{\hat{u}(\theta_j)^2}{\sin^2 \theta_j} \cdot N^{-1} \sin \frac{\theta_j}{2} \cos \frac{\theta_j}{2} \\
&\lesssim N^{-1} \sum_{j=1}^{N-1} \hat{u}^2(\theta_j) (\sin \theta_j)^{-1} \\
&\lesssim N^{-1} \sum_{j=1}^{N-1} \frac{1}{\sin \theta_j} \max_{\theta \in K_j} |\hat{u}(\theta)|^2,
\end{align*}
其中每个 $\theta_j$ 都位于某个区间 $K_j $ 内, $\theta_j \in K_j \subset [a_0, a_1] \subset (0, \pi)$,其中 $a_0 = \frac{O(1)}{N-1}$, $a_1 = \frac{N\pi + O(1)}{N-1}$, 并且$K_j$的区间长度等于 $\frac{1}{N-1}$.
由于
\begin{align*}
\max_{\theta \in K_j} |\hat{u}(\theta)| &\lesssim \sqrt{N-1} \| \hat{u} \|_{L^2(K_j)} + \frac{1}{\sqrt{N^{-1}}} \| \partial_\theta \hat{u} \|_{L^2(K_j)}, \\
\max_{\theta \in K_j} |\hat{u}(\theta)|^2 &\lesssim \left( \sqrt{N-1} \| \hat{u} \|_{L^2(K_j)} + \frac{1}{\sqrt{N-1}} \| \partial_\theta \hat{u} \|_{L^2(K_j)} \right)^2 \\
&\lesssim (N-1) \| \hat{u} \|_{L^2(K_j)}^2 + \frac{1}{N-1} \| \partial_\theta \hat{u} \|_{L^2(K_j)}^2,
\end{align*}
故
\begin{align*}
\| I_N u \|_{w^{-1, -1}}^2 &\lesssim \sum_{j=1}^{N-1} \frac{1}{\sin \theta_j} \| \hat{u} \|_{L^2(K_j)}^2 + \sum_{j=1}^{N-1} \frac{1}{\sin \theta_j} N^{-2} \| \partial_\theta \hat{u} \|_{L^2(K_j)}^2 \\
&\lesssim \sum_{j=1}^{N-1} \int_{K_j} \frac{|\hat{u}|^2}{\sin \theta} \, d\theta + N^{-2} \sum_{j=1}^{N-1} \int_{K_j} \frac{|\partial_\theta \hat{u}|^2}{\sin \theta} \, d\theta \\
&\lesssim \int_0^\pi \frac{|\hat{u}|^2}{\sin \theta} \, d\theta + N^{-2} \int_0^\pi \frac{|\partial_\theta \hat{u}|^2}{\sin \theta} \, d\theta,
\end{align*}
因为
\begin{align*}
\int_0^\pi \frac{|\hat{u}|^2}{\sin \theta} \, d\theta &\overset{x=-\cos\theta}{=\joinrel=\joinrel=\joinrel=} \int_{-1}^1 \frac{|u(x)|^2}{1-x^2} \, dx = \| u \|_{w^{-1, -1}}^2, \\
\int_0^\pi \frac{|\partial_\theta \hat{u}|^2}{\sin \theta} \, d\theta &\overset{x=-\cos\theta}{=\joinrel=\joinrel=\joinrel=} \int_{-1}^1 |\partial_x u|^2 \, dx = \| \partial_x u \|^2,
\end{align*}
故
$$\| I_N u \|_{w^{-1, -1}}^2 \lesssim \| u \|_{w^{-1, -1}}^2 + N^{-2} \| \partial_x u \|^2. $$
整理知原题结论成立.
\subsection*{4.(a)}
考虑如下两点边值问题:
\begin{equation}
    u(x) - u''(x) = f(x), \quad -1 < x < 1; \quad u(-1)=0, \quad u(1)=1.
    \label{eq:pde_4a}
\end{equation}
其中真解 $u(x)$ 定义为:
\begin{equation}
    u(x) = \begin{cases} 
        0, & x \in [-1, 0], \\
        x^\gamma, & x \in (0, 1].
    \end{cases}
\end{equation}
这里 $\gamma \in \{4, 5, 6\}$.右端项 $f(x)$ 由 $u(x)$ 导出.要求使用 Legendre-Galerkin 方法求解,取 $N = 2^i, i=4,\dots,9$,并分析误差 $\|u - u_N\|_{N,w}$ 的收敛性.

\textbf{1. 齐次化边界条件}

由于原问题边界条件非齐次,令数值解 $u_N(x)$ 为:
\begin{equation}
    u_N(x) = v_N(x) + u_b(x).
\end{equation}
其中 $u_b(x) = \frac{x+1}{2}$ 是满足边界条件的提升函数,$v_N(x)$ 满足齐次边界条件 $v_N(\pm 1) = 0$.将 $u_N$ 代入方程 \eqref{eq:pde_4a},并注意到 $u_b''(x)=0$,得到关于 $v_N$ 的方程:
\begin{equation}
    v_N - v_N'' = f - u_b.
\end{equation}

\textbf{2. 变分形式与离散化}

选取逼近空间 $V_N = \text{span}\{\phi_0, \dots, \phi_{N-2}\}$,其中
\begin{equation}
    \phi_k(x) = L_k(x) - L_{k+2}(x), \quad k=0, 1, \dots, N-2.
\end{equation}
易知 $\phi_k(\pm 1) = 0$.Legendre-Galerkin 方法的弱形式为:寻找 $v_N \in V_N$,使得对所有 $\phi_j \in V_N$:
\begin{equation}
    (v_N, \phi_j) + (v_N', \phi_j') = (I_N f - u_b, \phi_j).
\end{equation}
令 $v_N(x) = \sum_{k=0}^{N-2} \hat{v}_k \phi_k(x)$,代入上式得到线性方程组:
\begin{equation}
    (\mathbf{M} + \mathbf{S}) \mathbf{\hat{v}} = \mathbf{F}.
\end{equation}
其中:
\begin{itemize}
    \item 矩阵 $\mathbf{S}$ ($S_{jk} = (\phi_k', \phi_j')$) 为对角阵:
    \begin{equation}
        S_{jk} = (4k+6)\delta_{jk}.
    \end{equation}
    \item 矩阵 $\mathbf{M}$ ($M_{jk} = (\phi_k, \phi_j)$) 为对称五对角阵,非零元素为:
    \begin{equation}
        M_{kk} = \frac{2}{2k+1} + \frac{2}{2k+5}, \quad M_{k, k+2} = M_{k+2, k} = -\frac{2}{2k+5}.
    \end{equation}
    \item 向量 $\mathbf{F}$:
    \begin{equation}
        F_j = \int_{-1}^1 I_N f(x)\phi_j(x)dx - \int_{-1}^1 u_b(x)\phi_j(x)dx.
    \end{equation}
\end{itemize}

\textbf{3. 数值结果}

图 \ref{fig:convergence} 展示了 $\log_{10}\|u - u_N\|_{N,w}$ 随 $\log_{10} N$ 的变化关系.

谱方法的收敛速度通常取决于解的光滑性.对于本题,解析解 $u(x)$ 在 $x=0$ 处存在奇点.虽然 $u$ 连续,但在 $x=0$ 处的 $\gamma$ 阶导数不连续.具体来说,解属于 Sobolev 空间 $H^{\gamma + 1/2 - \epsilon}$.根据逼近理论,Legendre 谱方法的 $L^2$ 误差估计为:
\begin{equation}
    \|u - u_N\| \le C N^{-s} \|u\|_{H^s}.
\end{equation}
在此问题中,受限于 $x=0$ 处的正则性,预期的收敛阶为 $O(N^{-(\gamma + 0.5)})$. 数值实验结果(斜率约为 $-4.5, -5.5, -6.5$)与之吻合.

\begin{figure}[H]
    \centering
    % 请确保当前目录下有生成的图片文件,例如命名为 convergence_plot.png
    % 如果没有图片,请先运行代码生成或注释掉下面这行
    \includegraphics[width=0.85\textwidth]{figures/result_4a.png}
    \caption{Legendre-Galerkin 方法在不同 $\gamma$ 下的误差收敛曲线}
    \label{fig:convergence}
\end{figure}

\subsection*{4.(b)}
考虑如下变系数两点边值问题:
\begin{equation}
    x^2 u(x) - (e^x u'(x))' = f(x), \quad -1 < x < 1.
    \label{eq:pde_4b}
\end{equation}
其中真解 $u(x)$ 定义为:
\begin{equation}
    u(x) = \begin{cases} 
        0, & x \in [-1, 0], \\
        x^\gamma, & x \in (0, 1].
    \end{cases}
\end{equation}
考虑 $\gamma \in \{4, 5, 6\}$,并针对以下两种边界条件分别求解:
\begin{itemize}
    \item \textbf{情形 1:} $u(-1)=0, \quad u(1)=1$
    \item \textbf{情形 2:} $u(-1) - u'(-1)=0, \quad u(1)=1$
\end{itemize}
题目要求使用 Chebyshev 配点法(强形式)进行求解,取 $N = 2^i, i=4,\dots,9$,并分析误差 $\|u - u_N\|_{N,w}$ 的收敛性.

\textbf{1. 算子展开与离散}

首先将方程 \eqref{eq:pde_4b} 中的微分项展开:
\begin{equation}
    x^2 u(x) - \left( e^x u'(x) + e^x u''(x) \right) = f(x).
\end{equation}
即:
\begin{equation}
    x^2 u - e^x u' - e^x u'' = f.
    \label{eq:expanded}
\end{equation}

选取 Chebyshev-Gauss-Lobatto节点 $x_j = \cos(j\pi/N), j=0,\dots,N$. 记Chebyshev 微分矩阵为 $\mathbf{D}$.若 $\mathbf{u} = [u(x_0), \dots, u(x_N)]^T$,则节点处的一阶导数和二阶导数向量分别为 $\mathbf{u}' \approx \mathbf{D}\mathbf{u}$ 和 $\mathbf{u}'' \approx \mathbf{D}^2\mathbf{u}$.
将方程 \eqref{eq:expanded} 写成矩阵形式:
\begin{equation}
    (\mathbf{X}^2 - \mathbf{E}\mathbf{D} - \mathbf{E}\mathbf{D}^2) \mathbf{u} = \mathbf{f}.
\end{equation}
其中 $\mathbf{X}$ 和 $\mathbf{E}$ 分别为对角元是 $x_j$ 和 $e^{x_j}$ 的对角矩阵.记总系数矩阵为 $\mathbf{A}_{global}$,则有线性系统 $\mathbf{A}_{global} \mathbf{u} = \mathbf{f}$.

\textbf{2. 边界条件处理}

在强形式配点法中,边界条件通过直接替换矩阵 $\mathbf{A}_{global}$ 的对应行来施加.

\textbf{右边界 ($x_0 = 1$)}
条件为 $u(1)=1$.
在矩阵系统中,这是第一个未知量 $u_0$.我们将 $\mathbf{A}$ 的第 0 行替换为 $[1, 0, \dots, 0]$,并将右端项向量 $\mathbf{f}$ 的第 0 个元素设为 1.
\begin{equation}
    1 \cdot u_0 + 0 \cdot u_1 + \dots = 1.
\end{equation}

\textbf{情形 1:} $u(-1) = 0$.
类似地,将第 $N$ 行替换为 $[0, \dots, 0, 1]$,右端项设为 0.

\textbf{情形 2:} $u(-1) - u'(-1) = 0$.
这里涉及到导数.离散形式为:
\begin{equation}
    u_N - \sum_{k=0}^N D_{N,k} u_k = 0.
\end{equation}
写成向量形式为 $(\mathbf{e}_N^T - \mathbf{D}_{N,:}) \mathbf{u} = 0$,其中 $\mathbf{e}_N$ 是第 $N$ 个分量为 1 的单位向量.
我们在代码中将矩阵 $\mathbf{A}$ 的第 $N$ 行替换为向量:
\begin{equation}
    \mathbf{row}_N = [0, \dots, 0, 1] - [D_{N,0}, \dots, D_{N,N}].
\end{equation}
并将右端项 $f_N$ 设为 0.

\textbf{3. 数值结果}

利用 Python 代码计算并绘制了两种边界条件下,$\gamma=4,5,6$ 时的误差收敛曲线(图 \ref{fig:cheb_result}).

\begin{figure}[H]
    \centering
    % 注意:请确保运行代码生成了此图片,或者使用代码解释器生成的图片
    \includegraphics[width=\textwidth]{figures/result_4b.png} 
    \caption{Chebyshev 配点法在不同边界条件下的误差收敛曲线}
    \label{fig:cheb_result}
\end{figure}

其一, 代数收敛性. 结果显示误差呈代数收敛而非谱收敛(指数收敛).这是因为真解 $u(x)$ 在 $x=0$ 处存在奇异性.对于 Chebyshev 逼近,若函数 $u$ 的 $k$ 阶导数有界变差,则误差衰减率为 $O(N^{-k})$.
\begin{itemize}
        \item 当 $\gamma=4$ 时,$u^{(4)}(x)$ 在 $x=0$ 处不连续,故预期收敛阶为 4.实验结果吻合.
\item 当 $\gamma=6$ 时,$u^{(6)}(x)$ 在 $x=0$ 处不连续,预期收敛阶为 6.实验结果吻合.
    \end{itemize}

其二, $\gamma=5$ 的特殊情况. 理论上应观察到 5 阶收敛,但实验数据接近 4 阶.这可能是由于强形式配点法中引入了二阶微分矩阵 $\mathbf{D}^2$,其条件数随 $N$ 增长较快($O(N^4)$),加之奇点 $x=0$ 正好位于区间中央,导致数值误差的主导项发生变化.

其三, 边界条件的影响. 对比情形 1 和情形 2,收敛曲线的斜率和误差量级几乎一致.这说明通过“行替换”的方式处理 Robin 边界条件($u - u' = 0$)是非常有效的,不会破坏 Chebyshev 方法的精度.

\subsection*{5.(a)}
显然.

\subsection*{5.(b) 数值求解方案}

\subsubsection*{1. 算子分裂 (Operator Splitting)}

考虑 Fisher 方程 $\partial_t u = \partial_{xx} u + u(1-u)$.我们将其分裂为非线性反应部分 $\mathcal{R}$ 和线性扩散部分 $\mathcal{D}$:
\begin{align}
    \mathcal{R}: \quad & \partial_t u = u(1-u), \\
    \mathcal{D}: \quad & \partial_t u = \partial_{xx} u.
\end{align}
设时间步长为 $\tau$,从 $t_n$ 到 $t_{n+1}$ 采用二阶 Strang 分裂格式:
\begin{equation}
    u^{n+1} = \mathcal{S}_{\mathcal{R}}(\tau/2) \circ \mathcal{S}_{\mathcal{D}}(\tau) \circ \mathcal{S}_{\mathcal{R}}(\tau/2) u^n,
\end{equation}
其中 $\mathcal{S}_{\mathcal{R}}(t)$ 为反应步的精确解算子.对于初值问题 $\frac{\mathrm{d} u}{\mathrm{d} t} = u(1-u), \, u(0)=u_0$,其解析解为:
\begin{equation}
    u(t) = \frac{u_0}{u_0 + (1-u_0)\mathrm{e}^{-t}}.
\end{equation}

\subsubsection*{2. 空间离散与边界处理}

对于扩散步 $\partial_t u = \partial_{xx} u, \ x \in [-L, L]$,作坐标变换 $\xi = x/L, \ \xi \in [-1, 1]$,方程变为:
\begin{equation}
    \partial_t u = \frac{1}{L^2} \partial_{\xi\xi} u.
\end{equation}
为处理非齐次边界条件 $u(-1,t)=1, u(1,t)=0$,引入边界函数 $u_B(\xi) = \frac{1-\xi}{2}$.
令 $u(\xi, t) = v(\xi, t) + u_B(\xi)$,则 $v$ 满足齐次 Dirichlet 边界条件 $v(\pm 1, t) = 0$.
注意到 $\partial_{\xi\xi} u_B = 0$,故 $v$ 满足的方程仍为 $\partial_t v = \frac{1}{L^2} \partial_{\xi\xi} v$.

选取 Legendre-Galerkin 基函数 $\phi_k(\xi) = L_k(\xi) - L_{k+2}(\xi)$,将数值解展开为:
\begin{equation}
    v_N(\xi, t) = \sum_{k=0}^{N-2} \hat{v}_k(t) \phi_k(\xi).
\end{equation}
代入方程并作 Galerkin 投影,得到线性系统:
\begin{equation}
    \mathbf{M} \frac{\mathrm{d}\hat{\boldsymbol{v}}}{\mathrm{d}t} = -\frac{1}{L^2} \mathbf{S} \hat{\boldsymbol{v}},
\end{equation}
其中刚度矩阵与质量矩阵元素分别为:
\begin{equation}
    M_{kj} = (\phi_j, \phi_k)_{L^2}, \quad S_{kj} = (\phi'_j, \phi'_k)_{L^2}.
\end{equation}
利用 Crank-Nicolson 格式进行时间离散,需在每一步求解如下线性方程组:
\begin{equation}
    \left( \mathbf{M} + \frac{\tau}{2L^2}\mathbf{S} \right) \hat{\boldsymbol{v}}^{n+1} = \left( \mathbf{M} - \frac{\tau}{2L^2}\mathbf{S} \right) \hat{\boldsymbol{v}}^n.
\end{equation}
求解得到 $\hat{\boldsymbol{v}}^{n+1}$ 后,通过 $u^{n+1} = v_N^{n+1} + u_B$ 恢复物理空间解,再进行下一步的非线性更新.

\textbf{3. 数值结果}

取参数 $N=128, L=100, \tau=10^{-3}$.
数值解与精确解 $u_{exact}(x,t) = [1 + \exp((x/\sqrt{6}) - (5/6)t)]^{-2}$ 的最大误差 ($L^\infty$) 如下表所示:
\begin{table}[H]
    \centering
    \caption{Fisher 方程数值解与精确解的最大误差}
    \label{tab:fisher_error}
    \begin{tabular}{cc}
        \toprule
        Time ($t$) & Max Error ($L^\infty$) \\
        \midrule
        1 & $2.91 \times 10^{-5}$ \\
        2 & $1.49 \times 10^{-5}$ \\
        3 & $2.79 \times 10^{-5}$ \\
        4 & $4.81 \times 10^{-5}$ \\
        5 & $1.15 \times 10^{-4}$ \\
        6 & $3.21 \times 10^{-4}$ \\
        \bottomrule
    \end{tabular}
\end{table}

各时间点的数值解如图所示,波形由 $x \approx 0$ 处向右平稳传播,与 Fisher 方程行波解性质一致.
\begin{figure}[H]
    \centering
    \includegraphics[width=0.8\textwidth]{figures/fisher_result_b.png}
    \caption{Fisher 方程在 $t=1, \dots, 6$ 时刻的数值解}
\end{figure}

\subsection*{5.(c)}
结果如下图所示.
\begin{figure}[H]
    \centering
    \includegraphics[width=0.8\textwidth]{figures/fisher_result_c1.png}
    \caption{Fisher 方程在初值条件$u(x,0) = \frac{1}{2} \left[ 1 - \tanh(x) \right]$下的动力学}
\end{figure}
\begin{figure}[H]
    \centering
    \includegraphics[width=0.8\textwidth]{figures/fisher_result_c2.png}
    \caption{Fisher 方程在初值条件$u(x,0) = \frac{1}{\left[1+e^{(x+5)/2}\right]^{2}} \frac{1}{\left[1+e^{x/3}\right]^{2}} \frac{1}{\left[1+e^{(x-5)/4}\right]^{2}}$下的动力学}
\end{figure}

\subsection*{6.(a)}
考虑定义在 $\Omega = (-1, 1)^2$ 上的二维可分偏微分方程:
\begin{equation}
    a u_{xx} + b u_x + c u + d u_{yy} + e u_y + f u = g(x, y).
    \label{eq:pde_6a}
\end{equation}
边界条件为:
\begin{itemize}
    \item $x$ 方向 (Neumann): $\partial_x u(\pm 1, y) = 0, \quad y \in [-1, 1]$
    \item $y$ 方向 (Dirichlet): $u(x, \pm 1) = 0, \quad x \in [-1, 1]$
\end{itemize}
其中系数 $a, \dots, f$ 均为常数.

\textbf{1. 基函数的构造}

为了利用 Galerkin 方法,我们需要构造满足相应边界条件的基函数空间.令 $L_k(x)$ 为 $k$ 次 Legendre 多项式.

\textbf{$x$ 方向}:
   构造基函数 $\phi_k(x)$ 满足 $\phi_k'(\pm 1) = 0$:
   \begin{equation}
       \phi_k(x) = L_k(x) - \frac{k(k+1)}{(k+2)(k+3)} L_{k+2}(x), \quad k=0, 1, \dots, N-2.
   \end{equation}
   
\textbf{$y$ 方向}:
   构造基函数 $\psi_k(y)$ 满足 $\psi_k(\pm 1) = 0$:
   \begin{equation}
       \psi_k(y) = L_k(y) - L_{k+2}(y), \quad k=0, 1, \dots, N-2.
   \end{equation}

\textbf{2. 离散化格式}

设数值解 $u_N(x, y) = \sum_{i=0}^{N-2} \sum_{j=0}^{N-2} u_{ij} \phi_i(x) \psi_j(y)$.代入方程并对测试函数 $\phi_m(x) \psi_n(y)$ 做 Galerkin 投影,可得 Sylvester 矩阵方程:
\begin{equation}
    S_x U M_y^T + M_x U S_y^T = G.
\end{equation}
其中 $U$ 是待求系数矩阵, $G_{mn} = (g, \phi_m \psi_n)$, $(M_y)_{mn} = \int_{-1}^1 \psi_m(y) \psi_n(y) dy$, $(S_x)_{mn} = \int_{-1}^1 \left( a \phi_n''(x) + b \phi_n'(x) + c \phi_n(x) \right) \phi_m(x) dx$.

\textbf{3. 系数矩阵的计算}

易知
\begin{equation}
    \int_{-1}^1 L_m(x) L_n(x) dx = h_m \delta_{mn}, \quad \text{其中 } h_m = \frac{2}{2m+1}.
\end{equation}

故
\begin{align}
    (M_y)_{mn} &= \int_{-1}^1 \left( L_m(y) - L_{m+2}(y) \right) \left( L_n(y) - L_{n+2}(y) \right) dy \nonumber \\
    &= \int L_m L_n - \int L_m L_{n+2} - \int L_{m+2} L_n + \int L_{m+2} L_{n+2} \\
    &= h_m \delta_{mn} - h_m \delta_{m, n+2} - h_{m+2} \delta_{m+2, n} + h_{m+2} \delta_{m, n}.
\end{align}

选取 Legendre-Gauss 节点 $x_q$ 和对应权重 $w_q$, $(S_x)_{mn}$可被精确离散为有限和:
\begin{equation}
    (S_x)_{mn} = \sum_{q=0}^{Q-1} w_q \left[ a \phi_n''(x_q) + b \phi_n'(x_q) + c \phi_n(x_q) \right] \phi_m(x_q).
\end{equation}

\textbf{4. 结果分析}

为了验证算法的准确性和收敛阶,我们设计一个满足边界条件的解析解:
\begin{equation}
    u_{exact}(x, y) = \cos(\pi x) \cos\left(\frac{\pi}{2} y\right).
\end{equation}
该解满足 $x=\pm 1$ 处的 Neumann 条件(导数为 0)和 $y=\pm 1$ 处的 Dirichlet 条件(值为 0).令$a=1, b=0.5, c=1, d=1, e=-0.5, f=2$.
图 \ref{fig:convergence_a} 展示了随着多项式截断阶数 $N$ 的增加,最大误差 ($L^\infty$ error) 的变化情况.可以看出,误差在半对数坐标系下呈线性下降,体现了谱收敛特性.当 $N \ge 24$ 时,误差已达到机器精度 ($10^{-14}$ 量级).
\begin{figure}[H]
    \centering
    \includegraphics[width=0.7\textwidth]{figures/Q6a_error_convergence.png}
    \caption{误差随截断阶数 $N$ 的收敛曲线 (semilog坐标)}
    \label{fig:convergence_a}
\end{figure}

\subsection*{6.(b)}
\textbf{1. 基函数的构造}

基函数的构造与 6(a) 完全一致.

\textbf{2. 离散化格式}

设数值解 $u_N(x, y) = \sum_{i, j} u_{ij} \phi_i(x) \psi_j(y)$.尽管系数 $a,b,c$ 依赖于 $x$,但算子仍然可以分离为 $L = \mathcal{L}_x + \mathcal{L}_y$.Galerkin 投影后的矩阵方程形式依然为 Sylvester 方程:
\begin{equation}
    S_x U M_y^T + M_x U S_y^T = G.
\end{equation}
其中 $y$ 方向的矩阵 $M_y, S_y$ 与常系数情况完全相同.
主要区别在于 $x$ 方向的变系数刚度矩阵:
\begin{equation}
    (S_x)_{mi} = \int_{-1}^1 \left[ a(x)\phi_i''(x) + b(x)\phi_i'(x) + c(x)\phi_i(x) \right] \phi_m(x) dx.
\end{equation}

\textbf{3. 变系数矩阵的计算}

对于变系数积分,解析计算较为困难且不通用.我们采用 Gauss-Legendre 数值积分进行高精度近似.
选取 $Q$ 个积分节点 $x_q$ 和权重 $w_q$(取 $Q \ge N + k$,以处理系数函数的非线性),则 $(S_x)_{mi}$ 计算如下:
\begin{equation}
    (S_x)_{mi} \approx \sum_{q=0}^{Q-1} w_q \underbrace{\left[ a(x_q)\phi_i''(x_q) + b(x_q)\phi_i'(x_q) + c(x_q)\phi_i(x_q) \right]}_{\text{算子在节点上的值}} \phi_m(x_q).
\end{equation}

\textbf{4. 结果分析}

设定系数为:
$$ a(x) = 2 + \sin(x), \quad b(x) = x, \quad c(x) = e^{x/2}, d=e=f=1. $$
并构造精确解:
\begin{equation}
    u_{exact}(x, y) = \sin^2(\pi x) \cos\left(\frac{\pi}{2} y\right).
\end{equation}
验证可见:$\partial_x u|_{x=\pm 1} \propto \sin(2\pi)=0$, $u|_{y=\pm 1} \propto \cos(\pm \pi/2)=0$.
图 \ref{fig:convergence_b} 展示了误差收敛曲线.结果表明,即使引入变系数,Legendre-Galerkin 方法依然保持了谱收敛特性,误差随着 $N$ 的增加呈指数级衰减.

\begin{figure}[H]
    \centering
    \includegraphics[width=0.7\textwidth]{figures/Q6b_error_convergence.png}
    \caption{变系数方程误差收敛曲线 (semilog坐标)}
    \label{fig:convergence_b}
\end{figure}

\end{document}